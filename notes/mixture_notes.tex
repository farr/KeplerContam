\documentclass[modern]{aastex62}

\usepackage{amsmath}
\usepackage{amssymb}

\newcommand{\dd}{\mathrm{d}}
\newcommand{\diff}[2]{\frac{\dd #1}{\dd #2}}

\begin{document}

\title{Notes on the Model}
\author[0000-0003-1540-8562]{Will M. Farr}
\affiliation{Department of Physics and Astronomy, Stony Brook University, Stony Brook NY 11794, USA}
\affiliation{Center for Computational Astronomy, Flatiron Institute, 162 5th Ave., New York NY 10010, USA}
\email{will.farr@stonybrook.edu}

\begin{abstract}
  I discuss the contaminated mixture model that we are using.
\end{abstract}

\section{Hierarchical Likelihood}

We assume, following \citet{Foreman-Mackey2014}, that the occurrence rate of
planets is constant in some set of bins in the period-radius plane, $R_i \leq R
< R_{i+1}$, $P_i \leq P \leq P_{i+1}$, $i = 1, \ldots, N_\mathrm{bin}$:
%
\begin{equation}
  \diff{N}{\ln P \dd \ln R} = \begin{cases}
  n_i & R_i \leq R < R_{i+1} \textnormal{ and } P_i \leq P \leq P_{i+1} \\
  \tilde{n} & \textnormal{otherwise}
\end{cases}.
\end{equation}
%
We include a ``catch all'' bin, where the rate density is $\tilde{n}$ to avoid
issues at the edges of the set of bins; with the catch-all bin, we can cut the
data set several ``sigma'' outside the region of interest with bins, and objects
that land on the edges will be distributed between the bins and the
``catch-all'' bin.  Note that, for all practical purposes, Kepler measures the
period perfectly, so the only uncertainty in any of the measurements is the
radius.

We also assume that the contaminated background follows a similar
constant-in-bin model,
%
\begin{equation}
  \diff{N_{\mathrm{bg}}}{\ln P \dd \ln R} = b_i \quad R_i \leq R < R_{i+1} \textnormal{ and } P_i \leq P \leq P_{i+1},
\end{equation}
%
but here we do not include the extra bin (since we don't care about the
background-foreground distinction in that bin).

Let us assume that the likelihood is well-approximated by a delta function in
period and a log-normal distribution for the radius for each planet detection:
%
\begin{equation}
    p\left( d \mid \ln P, \ln R \right) \propto \delta\left( \ln P_\mathrm{obs} - \ln P \right) \frac{1}{\sqrt{2\pi} \sigma_{\ln R}} \exp\left[ \frac{\left( \ln R_\mathrm{obs} - \ln R \right)^2}{2 \sigma_{\ln R}^2} \right].
\end{equation}
%
Then the hierarchical likelihood for a set of planet candidates can be obtained
by integrating out the ``nuisance'' parameters giving the true period and radius
of each planet \citep{Mandel2018,Loredo2004}:
%
\begin{multline}
  \label{eq:hier-likelihood}
  p\left( \left\{ d_j \right\} \mid \left\{ n_i \right\}, \tilde{n}, \left\{b_i \right\} \right) \propto  \left[ \prod_{j = 1}^{N} \int \dd \ln P_j \, \dd \ln R_j \, p\left( d_j \mid \ln P_j, \ln R_j \right) \right. \\ \times \left. \left( \diff{N}{\ln P_j \dd \ln R_j} + \diff{N_\mathrm{bg}}{\ln P_j \dd \ln R_j} \right) \right] \exp\left[ - N_\mathrm{ex} - N_\mathrm{bg} - \tilde{N} \right];
\end{multline}
%
where $N_\mathrm{ex}$ is the expected number of true planet detections within the
grid of cells,
%
\begin{equation}
  N_\mathrm{ex} = \sum_{i=1}^{N_\mathrm{bin}} n_i \Delta \ln P_i \Delta \ln R_i \sum_{k = 1}^{N_\mathrm{stars}}  \left\langle P_\mathrm{det}^{(k)} \right\rangle_i,
\end{equation}
%
where $\left\langle P_\mathrm{det}^{(k)} \right\rangle_i$ is the detection
probability for a planet around star $k$ averaged over cell $i$ for a flat
population in $\ln P$ and $\ln R$; $N_\mathrm{bg}$ is the expected number of
background observations
%
\begin{equation}
  N_\mathrm{bg} = \sum_{i=1}^{N_\mathrm{bin}} b_i \Delta \ln P_i \Delta \ln R_i;
\end{equation}
%
and $\tilde{N}$ is the expected number of detections in the ``catch all'' bin
%
\begin{equation}
  \tilde{N} = \tilde{n} \tilde{V},
\end{equation}
%
where $\tilde{V}$ is the volume of the catch-all in $\ln P$-$\ln R$ space.
Because the population is assumed to be constant in bins, the integrals in Eq.
\eqref{eq:hier-likelihood} reduce to
%
\begin{equation}
  \label{eq:reduced-likelihood}
  p\left( \left\{ d_j \right\} \mid \left\{ n_i \right\}, \tilde{n}, \left\{b_i \right\} \right) \propto  \left[ \prod_{j = 1}^{N} \tilde{w}_j \tilde{n} + \sum_{i=1}^{N_\mathrm{bin}} w_{ji} \left( n_i + b_i \right) \right] \exp\left[ - N_\mathrm{ex} - N_\mathrm{bg} - \tilde{N} \right],
\end{equation}
%
where
%
\begin{equation}
  w_{ji} = \int_{R_i \leq R_j < R_{i+1} \cap P_i \leq P_j < P_{i+1}} \dd \ln P_j \, \dd \ln R_j \, p\left( d_j \mid \ln P_j, \ln R_j \right),
\end{equation}
%
are weights assigned to each bin (similarly for $\tilde{w}_j$ for the catch-all
bin).  The $w_{ji}$ can be pre-computed, as they depend only on the bin
boundaries and the observed periods, radii, and radius uncertainties.

\section{Prior}

The foreground and background per-bin rates, $n_i$ and $b_i$, are degenerate in
the likelihood in Eq.\ \eqref{eq:reduced-likelihood}.  We can break the
degeneracy using an informative prior on one or both rate distributions.  Here
we will assume that we have \emph{measured} the background contamination rate
(perhaps imperfectly) through some independent process, and incorporate this
through an informative prior on the background rate densities $b_i$.

Following \citet{Formean-Mackey2014}, we choose to estimate the background $b_i$
by using the results of a search over \emph{inverted} Kepler lightcurves using
the same pipelines used to produce the catalog \citep{Coughlin2017}.  TODO: more
details here.  We supply a log-normal prior on $b_i$ based on the observed count
of non \texttt{FP} inverted-lightcurve candidate events and a relative
uncertainty of $1/\sqrt{N}$.

We choose to implement a Gaussian process prior on the (log of the) $n_i$.  We
choose a constant-mean, squared exponential covariance GP on the log-rates:
%
\begin{equation}
  \ln \vec{n} \sim N\left( \mu \vec{1} , \mathbf{\Sigma} \right),
\end{equation}
%
with
%
\begin{equation}
  \Sigma_{ij} = \sigma^2 \left( 1 + \epsilon \delta_{ij} \right) \exp\left[ -\frac{\left| \Delta \vec{x}_{ij} \right|^2}{2 \lambda^2} \right].
\end{equation}
%
The single-bin variance is $\sigma^2 \left( 1 + \epsilon \right)$, $\epsilon \ll
1$ is a fractional white-noise component added to the diagonal terms of the
matrix for stability, $\Delta \vec{x}_{ij}$ is the displacement vector between
the centers of bin $i$ and bin $j$ in the $\ln P$-$\ln R$ plane, and $\lambda$
is a correlation length scale in this plane\footnote{A natural extension would
be to allow the $\ln P$ and $\ln R$ dimensions to have \emph{independent} length
scales; or even to impose an arbitrary \emph{metric} matrix that would allow for
arbitrary anisotropy in the correlations.}.

\section{Correlations in Planet Detection}

The model currently assumes that the detection probability for each planet is
independent of all the other planets in the same system.  This will
over-estimate the number of planets per star (since it almost certainly makes
second and subsequent planets seem harder to detect than they should be---at
least if there are \emph{any} planetary disks out there).  I will think about
how to modify the detection probability calculation to better reflect the fact
that planets occur in disks.  That, in itself, would be worth another paper; and
a particularly exciting one if we could find evidence for small disk opening
angles (i.e.\ once one planet transits, all the planets transit).

\section{Actual Data}

Plots of the actual data go here.

\bibliography{mixture_notes}

\end{document}
